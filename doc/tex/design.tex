\chapter{Design of Model and Solver}
\label{cha:design}

\section{Introduction}
Here we present the design of the model and solver parts. 



\section{Overall model}
The planner can be devided into three areas: the objects describing
the world, the solvers (planning algorithms) and the heuristics used
by the \astar solver to estimate the performance of a given state.

\section{World objects}
Each level the planner is to solve is given in a level format
described by \citet{sokobano:level_format}. This level is loaded into
an array of Strings each letter being one square in the Sokobano
world. These strings are mainly used when the world is being explored
by the algorithm, but in order to do some more advanced moves the
positions of the boxes, the player and the goals is stored in a set
giving access to the positions.  When performing an action this is
done by the action itself. The action is responsible for updating the
world according to what the effects of the action is.

The result of solving a levels returns an action trace that is also
defined by \citet{sokobano:level_format}. The length of this path is 
a good performance indicator.

\section{The heuristics}
The heuristics are thoroughly described in chapter
\ref{cha:heuristics}.

\section{Solvers}
The planner has two solvers: a BreadthFirst (BF) Solver and an \astar
solver. At a given state the solvers look at all neighbour states,
states reachable using a single action, and adds them to a queue of
states to investigate.

\subsection{BreadthFirst Solver}
The BF solver adds the new found states to the end of the queue,
called unexplored states. This ensures that states discovered first
are explored before states found later. As a result it finds the
shortest possible solution, but it might take very long as it does not
care if states are close to or far from the goal state.

When a state is found is added to unexplored states only if it has not
been seen before. When it is first found it is added to the set
discovered states which is then able to tell if a state has been found
before.

\subsection{\astar solver}
The \astar solver works very much like the BF solver though instead of
a FIFO (first in, first out) queue it uses a priority queue sorting
the unexplored states according to their performance estimations
(determined by the heuristic). If a heuristic always returning the
same number the solver works as a BF solver because the only number
determining the priority of a state is the number of steps to get to
the state.

As some of the heuristics might throw a DeadlockException the \astar
solver also has to be able to handle this. If such an exception is
caught, the state is added to discovered states without being added to
the unexplored states. In this way the neighbour states to the
deadlocked state is not being explored making a solution building on a
deadlocked state (that we are able to find) impossible.

% Solver using A* (problem step 3)
\section{User Interface}
We have found a 3D engine that could be used to display the results of
the solver activity. (See \url{http://sokobano.sourceforge.net/})
Therefore we chose implement the solvers and models described in as a
subpackage to this project, in order to let easily hack into the
controls. To be precise all code related to this project will be in
the package \texttt{gdi1sokoban.planning} and below. The rest of the
code is available via the GNU General Public License \cite{gpl}, under
whose term we also release our modifications and additions.
